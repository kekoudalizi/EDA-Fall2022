% Options for packages loaded elsewhere
\PassOptionsToPackage{unicode}{hyperref}
\PassOptionsToPackage{hyphens}{url}
%
\documentclass[
]{article}
\usepackage{amsmath,amssymb}
\usepackage{lmodern}
\usepackage{iftex}
\ifPDFTeX
  \usepackage[T1]{fontenc}
  \usepackage[utf8]{inputenc}
  \usepackage{textcomp} % provide euro and other symbols
\else % if luatex or xetex
  \usepackage{unicode-math}
  \defaultfontfeatures{Scale=MatchLowercase}
  \defaultfontfeatures[\rmfamily]{Ligatures=TeX,Scale=1}
\fi
% Use upquote if available, for straight quotes in verbatim environments
\IfFileExists{upquote.sty}{\usepackage{upquote}}{}
\IfFileExists{microtype.sty}{% use microtype if available
  \usepackage[]{microtype}
  \UseMicrotypeSet[protrusion]{basicmath} % disable protrusion for tt fonts
}{}
\makeatletter
\@ifundefined{KOMAClassName}{% if non-KOMA class
  \IfFileExists{parskip.sty}{%
    \usepackage{parskip}
  }{% else
    \setlength{\parindent}{0pt}
    \setlength{\parskip}{6pt plus 2pt minus 1pt}}
}{% if KOMA class
  \KOMAoptions{parskip=half}}
\makeatother
\usepackage{xcolor}
\usepackage[margin=2.54cm]{geometry}
\usepackage{color}
\usepackage{fancyvrb}
\newcommand{\VerbBar}{|}
\newcommand{\VERB}{\Verb[commandchars=\\\{\}]}
\DefineVerbatimEnvironment{Highlighting}{Verbatim}{commandchars=\\\{\}}
% Add ',fontsize=\small' for more characters per line
\usepackage{framed}
\definecolor{shadecolor}{RGB}{248,248,248}
\newenvironment{Shaded}{\begin{snugshade}}{\end{snugshade}}
\newcommand{\AlertTok}[1]{\textcolor[rgb]{0.94,0.16,0.16}{#1}}
\newcommand{\AnnotationTok}[1]{\textcolor[rgb]{0.56,0.35,0.01}{\textbf{\textit{#1}}}}
\newcommand{\AttributeTok}[1]{\textcolor[rgb]{0.77,0.63,0.00}{#1}}
\newcommand{\BaseNTok}[1]{\textcolor[rgb]{0.00,0.00,0.81}{#1}}
\newcommand{\BuiltInTok}[1]{#1}
\newcommand{\CharTok}[1]{\textcolor[rgb]{0.31,0.60,0.02}{#1}}
\newcommand{\CommentTok}[1]{\textcolor[rgb]{0.56,0.35,0.01}{\textit{#1}}}
\newcommand{\CommentVarTok}[1]{\textcolor[rgb]{0.56,0.35,0.01}{\textbf{\textit{#1}}}}
\newcommand{\ConstantTok}[1]{\textcolor[rgb]{0.00,0.00,0.00}{#1}}
\newcommand{\ControlFlowTok}[1]{\textcolor[rgb]{0.13,0.29,0.53}{\textbf{#1}}}
\newcommand{\DataTypeTok}[1]{\textcolor[rgb]{0.13,0.29,0.53}{#1}}
\newcommand{\DecValTok}[1]{\textcolor[rgb]{0.00,0.00,0.81}{#1}}
\newcommand{\DocumentationTok}[1]{\textcolor[rgb]{0.56,0.35,0.01}{\textbf{\textit{#1}}}}
\newcommand{\ErrorTok}[1]{\textcolor[rgb]{0.64,0.00,0.00}{\textbf{#1}}}
\newcommand{\ExtensionTok}[1]{#1}
\newcommand{\FloatTok}[1]{\textcolor[rgb]{0.00,0.00,0.81}{#1}}
\newcommand{\FunctionTok}[1]{\textcolor[rgb]{0.00,0.00,0.00}{#1}}
\newcommand{\ImportTok}[1]{#1}
\newcommand{\InformationTok}[1]{\textcolor[rgb]{0.56,0.35,0.01}{\textbf{\textit{#1}}}}
\newcommand{\KeywordTok}[1]{\textcolor[rgb]{0.13,0.29,0.53}{\textbf{#1}}}
\newcommand{\NormalTok}[1]{#1}
\newcommand{\OperatorTok}[1]{\textcolor[rgb]{0.81,0.36,0.00}{\textbf{#1}}}
\newcommand{\OtherTok}[1]{\textcolor[rgb]{0.56,0.35,0.01}{#1}}
\newcommand{\PreprocessorTok}[1]{\textcolor[rgb]{0.56,0.35,0.01}{\textit{#1}}}
\newcommand{\RegionMarkerTok}[1]{#1}
\newcommand{\SpecialCharTok}[1]{\textcolor[rgb]{0.00,0.00,0.00}{#1}}
\newcommand{\SpecialStringTok}[1]{\textcolor[rgb]{0.31,0.60,0.02}{#1}}
\newcommand{\StringTok}[1]{\textcolor[rgb]{0.31,0.60,0.02}{#1}}
\newcommand{\VariableTok}[1]{\textcolor[rgb]{0.00,0.00,0.00}{#1}}
\newcommand{\VerbatimStringTok}[1]{\textcolor[rgb]{0.31,0.60,0.02}{#1}}
\newcommand{\WarningTok}[1]{\textcolor[rgb]{0.56,0.35,0.01}{\textbf{\textit{#1}}}}
\usepackage{graphicx}
\makeatletter
\def\maxwidth{\ifdim\Gin@nat@width>\linewidth\linewidth\else\Gin@nat@width\fi}
\def\maxheight{\ifdim\Gin@nat@height>\textheight\textheight\else\Gin@nat@height\fi}
\makeatother
% Scale images if necessary, so that they will not overflow the page
% margins by default, and it is still possible to overwrite the defaults
% using explicit options in \includegraphics[width, height, ...]{}
\setkeys{Gin}{width=\maxwidth,height=\maxheight,keepaspectratio}
% Set default figure placement to htbp
\makeatletter
\def\fps@figure{htbp}
\makeatother
\setlength{\emergencystretch}{3em} % prevent overfull lines
\providecommand{\tightlist}{%
  \setlength{\itemsep}{0pt}\setlength{\parskip}{0pt}}
\setcounter{secnumdepth}{-\maxdimen} % remove section numbering
\ifLuaTeX
  \usepackage{selnolig}  % disable illegal ligatures
\fi
\IfFileExists{bookmark.sty}{\usepackage{bookmark}}{\usepackage{hyperref}}
\IfFileExists{xurl.sty}{\usepackage{xurl}}{} % add URL line breaks if available
\urlstyle{same} % disable monospaced font for URLs
\hypersetup{
  pdftitle={Assignment 2: Coding Basics},
  pdfauthor={Jiawei Liang},
  hidelinks,
  pdfcreator={LaTeX via pandoc}}

\title{Assignment 2: Coding Basics}
\author{Jiawei Liang}
\date{}

\begin{document}
\maketitle

\hypertarget{directions}{%
\subsection{Directions}\label{directions}}

\begin{enumerate}
\def\labelenumi{\arabic{enumi}.}
\tightlist
\item
  Rename this file
  \texttt{\textless{}FirstLast\textgreater{}\_A02\_CodingBasics.Rmd}
  (replacing \texttt{\textless{}FirstLast\textgreater{}} with your first
  and last name).
\item
  Change ``Student Name'' on line 3 (above) with your name.
\item
  Work through the steps, \textbf{creating code and output} that fulfill
  each instruction.
\item
  Be sure to \textbf{answer the questions} in this assignment document.
\item
  When you have completed the assignment, \textbf{Knit} the text and
  code into a single PDF file.
\item
  After Knitting, submit the completed exercise (PDF file) to Sakai.
\end{enumerate}

\hypertarget{basics-day-1}{%
\subsection{Basics Day 1}\label{basics-day-1}}

\begin{enumerate}
\def\labelenumi{\arabic{enumi}.}
\item
  Generate a sequence of numbers from one to 100, increasing by fours.
  Assign this sequence a name.
\item
  Compute the mean and median of this sequence.
\item
  Ask R to determine whether the mean is greater than the median.
\item
  Insert comments in your code to describe what you are doing.
\end{enumerate}

\begin{Shaded}
\begin{Highlighting}[]
\CommentTok{\#1.Generate a sequence from 1 to 100,which increases by 4.}
\FunctionTok{seq}\NormalTok{(}\DecValTok{1}\NormalTok{,}\DecValTok{100}\NormalTok{, }\AttributeTok{by=}\DecValTok{4}\NormalTok{) }
\end{Highlighting}
\end{Shaded}

\begin{verbatim}
##  [1]  1  5  9 13 17 21 25 29 33 37 41 45 49 53 57 61 65 69 73 77 81 85 89 93 97
\end{verbatim}

\begin{Shaded}
\begin{Highlighting}[]
\CommentTok{\#2.Name this sequence \textquotesingle{}squence\_1\_to\_100\textquotesingle{}.}
\NormalTok{squence\_1\_to\_100}\OtherTok{\textless{}{-}}\FunctionTok{seq}\NormalTok{(}\DecValTok{1}\NormalTok{,}\DecValTok{100}\NormalTok{,}\AttributeTok{by=}\DecValTok{4}\NormalTok{) }
\CommentTok{\#3. compute the mean and median of this sequence.}
\FunctionTok{mean}\NormalTok{(squence\_1\_to\_100)  }
\end{Highlighting}
\end{Shaded}

\begin{verbatim}
## [1] 49
\end{verbatim}

\begin{Shaded}
\begin{Highlighting}[]
\FunctionTok{median}\NormalTok{(squence\_1\_to\_100) }
\end{Highlighting}
\end{Shaded}

\begin{verbatim}
## [1] 49
\end{verbatim}

\begin{Shaded}
\begin{Highlighting}[]
\CommentTok{\#4.Ask R whether mean is greater than median}
\FunctionTok{mean}\NormalTok{(squence\_1\_to\_100)}\SpecialCharTok{\textgreater{}}\FunctionTok{median}\NormalTok{(squence\_1\_to\_100)}
\end{Highlighting}
\end{Shaded}

\begin{verbatim}
## [1] FALSE
\end{verbatim}

\hypertarget{basics-day-2}{%
\subsection{Basics Day 2}\label{basics-day-2}}

\begin{enumerate}
\def\labelenumi{\arabic{enumi}.}
\setcounter{enumi}{4}
\item
  Create a series of vectors, each with four components, consisting of
  (a) names of students, (b) test scores out of a total 100 points, and
  (c) whether or not they have passed the test (TRUE or FALSE) with a
  passing grade of 50.
\item
  Label each vector with a comment on what type of vector it is.
\item
  Combine each of the vectors into a data frame. Assign the data frame
  an informative name.
\item
  Label the columns of your data frame with informative titles.
\end{enumerate}

\begin{Shaded}
\begin{Highlighting}[]
\CommentTok{\#1.Vector of names}
\NormalTok{aaa }\OtherTok{\textless{}{-}} \FunctionTok{c}\NormalTok{(}\StringTok{"Albert"}\NormalTok{,}\StringTok{"Bill"}\NormalTok{,}\StringTok{"Carl"}\NormalTok{,}\StringTok{"David"}\NormalTok{)}
\CommentTok{\#2.Vector of grades}
\NormalTok{bb }\OtherTok{\textless{}{-}} \FunctionTok{c}\NormalTok{(}\DecValTok{95}\NormalTok{, }\DecValTok{72}\NormalTok{, }\DecValTok{48}\NormalTok{, }\DecValTok{85}\NormalTok{)}
\CommentTok{\#3.Vector of whether or not they have passed the test}
\NormalTok{ccc }\OtherTok{\textless{}{-}} \FunctionTok{c}\NormalTok{(}\StringTok{\textquotesingle{}TRUE\textquotesingle{}}\NormalTok{,}\StringTok{\textquotesingle{}TRUE\textquotesingle{}}\NormalTok{,}\StringTok{\textquotesingle{}FALSE\textquotesingle{}}\NormalTok{,}\StringTok{\textquotesingle{}TRUE\textquotesingle{}}\NormalTok{)}
\CommentTok{\#4.Combine vectors into a data frame and call it Student\_transcript}
\NormalTok{Studentstranscript }\OtherTok{\textless{}{-}} \FunctionTok{data.frame}\NormalTok{(aaa,bb,ccc) }
\CommentTok{\#5.label the columns of my data \textquotesingle{}Name\textquotesingle{} \textquotesingle{}Scores\textquotesingle{} \textquotesingle{}Pass\textquotesingle{}}
\FunctionTok{names}\NormalTok{(Studentstranscript) }\OtherTok{\textless{}{-}} \FunctionTok{c}\NormalTok{(}\StringTok{"Name"}\NormalTok{,}\StringTok{"Scores"}\NormalTok{,}\StringTok{"Pass"}\NormalTok{);}
\end{Highlighting}
\end{Shaded}

9.QUESTION: How is this data frame different from a matrix?

\begin{quote}
Answer Things that they contain are different. A data frame can contain
characters, numbers, factors and times all at once. It could contains
different things. A matrix can only contain a single type.
\end{quote}

10.Create a function with an if/else statement. Your function should
take a \textbf{vector} of test scores and print (not return) whether a
given test score is a passing grade of 50 or above (TRUE or FALSE). You
will need to choose either the \texttt{if} and \texttt{else} statements
or the \texttt{ifelse} statement.

11.Apply your function to the vector with test scores that you created
in number 5.

\begin{Shaded}
\begin{Highlighting}[]
\FunctionTok{ifelse}\NormalTok{(bb }\SpecialCharTok{\textgreater{}=} \DecValTok{50}\NormalTok{,}\StringTok{\textquotesingle{}TRUE\textquotesingle{}}\NormalTok{,}\StringTok{\textquotesingle{}FALSE\textquotesingle{}}\NormalTok{)}
\end{Highlighting}
\end{Shaded}

\begin{verbatim}
## [1] "TRUE"  "TRUE"  "FALSE" "TRUE"
\end{verbatim}

\begin{enumerate}
\def\labelenumi{\arabic{enumi}.}
\setcounter{enumi}{11}
\tightlist
\item
  QUESTION: Which option of \texttt{if} and \texttt{else}
  vs.~\texttt{ifelse} worked? Why? \textgreater{} Answer `ifelse'works.
  For 'if' and `else',it checks whether a logical condition
  (i.e.~b\textgreater=50) is TRUE. If the logical condition is not TRUE,
  apply the content within the else statement (i.e.~return the sentence
  ``If condition was FALSE''). For `ifelse',it checks the logical
  condition we want to test. What should happen in case the logical
  condition is TRUE. What should happen in case the logical condition is
  FALSE. In this question, we need to check the result after the logical
  is true or false. So ifelse is better.
\end{enumerate}

\end{document}
